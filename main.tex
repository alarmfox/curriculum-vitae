\documentclass[11pt,a4paper]{article}

\usepackage{titlesec}
\usepackage{titling}
\usepackage{tikz}
\usepackage[margin=0.4in]{geometry}
\usepackage{graphicx}
\usepackage{xcolor}
\usepackage{xhfill}
\usepackage{hyperref}
\usepackage{fontawesome5}
\pagenumbering{gobble}

\input{glyphtounicode}

\pdfgentounicode=1

\definecolor{Purple}{HTML}{5F3772}
\newcommand\ruleafter[1]{#1~\xrfill[.7ex]{0.5pt}}
\renewcommand{\baselinestretch}{.9} 

\titleformat{\section}
{\Large\bfseries}
{}
{0em}
{}[{\color{Purple}\titlerule[1pt]}]

\titleformat{\subsection}
{\large\bfseries}
{}
{0em}
{\color{Purple}\ruleafter}

\titlespacing*{\subsection}{0pt}{2.5ex plus 1ex minus .2ex}{0.75em}

\titleformat{\subsubsection}[runin]
{\bfseries}
{}
{0em}
{}

\titlespacing*{\subsubsection}{0pt}{1.2ex plus .8ex minus .2ex}{1em}

\titleformat{\paragraph}[runin]
{\color{darkgray}\bf}
{}
{1em}
{}
\titlespacing*{\paragraph}{.5em}{1.7ex plus .2ex}{2em}


\renewcommand{\maketitle}
{

  \begin{minipage}[t]{0.70\textwidth}\vspace{-10pt}
    {\huge\bfseries
    \theauthor}

    \vspace{.40em}

    \faAt\  \href{mailto:capassog97@gmail.com}{capassog97@gmail.com}\newline
    \faLinkedin\  \href{https://www.linkedin.com/in/giuseppe-capasso97/}{linkedin.com/giuseppe-capasso} \newline
    \faGithub\  \href{https://github.com/alarmfox}{github.com/alarmfox} \newline
    \faGit\  \href{https://git.capass.org/}{git.capass.org}
  \end{minipage}
  \hfill
  \begin{minipage}[t]{0.30\textwidth}\vspace{-10pt}
    \begin{tikzpicture}
      \definecolor{Purple}{HTML}{5F3772}
      \clip (0,0) circle (2cm); 
      \node [anchor=center] at (.1,0) {\includegraphics[width=12em] {./images/propic.jpg}};
      \draw[Purple, line width=1.5mm] (0,0) circle (2cm);
    \end{tikzpicture}
  \end{minipage}

  }
  \begin{document}

  \title{Resume}

  \author{Giuseppe Capasso}
  \maketitle

  \section{Education}
  \subsection{University Federico II, Naples}
  \subsubsection{Master's degree in Computer Engineering\hfill \textcolor{darkgray}{\small{2022-now}}}
  \paragraph{Curriculum} Hybrid curriculum in cybersecurity, embedded systems and computer networks
  
  \subsubsection{Bachelor's degree in Computer Engineering\hfill \textcolor{darkgray}{\small{2016-2022}}}
  \paragraph{Curriculum} Multimedia Systems, Artificial Intelligence and Real Time Operating Systems 

  \subsubsection{Leonardo Cybersecurity Summer School\hfill \textcolor{darkgray}{\small{2024}}}
  \paragraph{Course contents} Activities focus on:
  \begin{itemize}
    \item Cloud security using Azure and Bicep scripts 
    \item Vulnerability assesment and penetration testing
    \item Cyber resilience concepts
  \end{itemize}

  \subsection{Cisco Networking Academy}
  \subsubsection{Cisco Devnet\hfill \textcolor{darkgray}{\small{2021}}} 
  \paragraph{Contents}Deep understanding of DevOps technologies like Chef, Ansible and Puppet. Network programmbility with Yang Models, RESTconf and NETconf. Infrastructure concepts with Docker containers and Jenkins focusing on cybersecurity.
  \subsubsection{Cisco DTLab Bootcamp\hfill \textcolor{darkgray}{\small{2019}}} 
  \paragraph{Contents} The bootcamp focused on giving strong networking fundamentals achieving \textbf{CCENT} and \textbf{PCAP} certifications. During the final phase, I worked in a team for \href{https://ssip.it/}{\textit{Italian Leather Research Institute}} and we built a remote collaboration system enhancing Cisco Webex experience with augmented reality.

  \section*{Work experience}
  \subsection{Consorzio Clara}
  \subsubsection{Mentor@Cisco DTLab Networking Academy\hfill \textcolor{darkgray}{\small{2021-now}}}

  \begin{itemize}
    \item Design and delivery of interactive activities for students on DevOps and Cybersecurity
    \item Manage network and server infrastructure for the laboratory
    \item Support and mentoring during challange and planning phases of the Bootcamp
    \item Bridging companies and students during project works phase
    \item Held workshops about containerzation, web concepts, REST APIs and cybersecurity 
  \end{itemize}

  \subsection{Gematica SRL}
  \subsubsection{Software engineer\hfill \textcolor{darkgray}{\small{2021-2023}}}
  \begin{itemize}
    \item Design and implementation of an air quality monitoring platform working with LoRaWAN and MQTT sensors. Technologies included Cassandra, Node.js, Python, Go-lang, embedded C, Kafka. Activities included managing a small frontend team
    \item Managed internal Cisco networking infrastructure implementing WPA2/Enteriprise policies using Packetfence to integrate Active Directory for employees and a splash portal for guests
    \item Design and implementation of passenger flow analysis in a metropolitan environment using Cisco WLC and SNMP traps
    \item Integrated with a circuit breaker using modbus TCP and GO
  \end{itemize}

  \subsection{NewNet}
  \subsubsection{Software developer \hfill \textcolor{darkgray}{\small{2019-2020}}}
  \begin{itemize}
    \item Developed a Desktop application using .NET Framework to manage company products'  software license
    \item Developed of a Mobile application using Xamarin for Zebra devices
    \item Implemented Aruba protocol for PDFs digital sign with a smart card using ASP.NET
  \end{itemize}

  \section{Projects}
  \subsection{Enclave benchmark \href{https://github.com/alarmfox/enclave-benchmark} {\faGithub}}
  A CLI application to run benchmarks and provide insightful metrics on Intel based Trusted Execution Environments (TEE) using \textit{Gramine}. The application collects low level metrics using \textit{perf}, \textit{Linux eBPF} technology and Intel RAPL for energy mmeasurement. \\

  \textbf{Keywords:} Rust, Linux eBPF, performance monitoring, Confidential Computing, Intel SGX

  \subsection{linux-av \href{https://github.com/alarmfox/linux-av}{\faGithub}}
  A malware scanner for Linux systems supporting rules based scanning and signature based scanning. The main feature is a \textbf{sanboxed} execution through Linux namespaces with \textit{system-call} monitoring using \textbf{eBPF} programs.\\

  \textbf{Keywords:} Rust, Kernel programming, Yara rules 
  
  \subsection{RT-Jam \href{https://github.com/alarmfox/rt-jam}{\faGithub}}
  A fully functional low latency audio streaming web platform using \textbf{QUIC/http3} protocol. The application is entirely written in Rust leveraging WebAssembly technology and uses \textit{Nats} and \textit{Protocol Buffers} for high performance media packet serialization and transmission. Furthermore, user authentication and email verification are supported. \\

  \textbf{Keywords:} Rust, WebAssembly, QUIC, HTTP3, Web development, Nats, Protocol buffer, Docker 
  
  \subsection{Game Repository \href{https://github.com/alarmfox/game-repository}{\faGithub} \href{https://game-repository.capass.org}{\faLaptopCode}}
  \textbf{\href{https://enactest-project.eu/}{Enactest}} aims to improve software testing skills through gamification. My team designed and developed a repository to store game data and make it available to other groups through a restful API. We also focused on automate deployment and test operations through github actions. \\

  \textbf{Keywords:} Go, REST API, OpenAPI, PostgreSQL, Docker, Github Actions, CI/CD 

  \subsection{Criticality Aware Kubernetes \href{https://github.com/alarmfox/criticality-aware-kubernetes}{\faGithub}}
  A deep analysis of inner \textbf{Kubernetes} logic to provide an upper bound to pod startup enabling support for 
  soft real-time applications in a cloud environment. After a cluster setup, I perfomed a bottlneck analysis to identify and validate latency sources with Extreme Value Theory models. I proposed a theoretical model for a schedulability test within mixed criticality systems. \\

  \textbf{Keywords:} Kubernetes, GO, Containers, Linux kernel, Python, Jupyter Notebook, Data analysis

  \subsection{Linux kernel fuzzing \href{https://github.com/alarmfox/progetto-software-security}{\faGithub}}
  I used Syzkaller to fuzz Linux IPsec implementation going through Netlink and Xfrm subsystems. Since IPsec is a 
  stateful protocol, I implemented \textbf{pseudo-syscalls} to setup kernel data-structures to guide coverage based 
  fuzzing. \\

  \textbf{Keywords:} Kernel fuzzing, Cybersecurity, IPsec, Syzkaller 

  \subsection{capass.org \href{https://github.com/alarmfox/personal-cloud}{\faGithub} \href{https://git.capass.org}{\faGit} \href{https://cloud.capass.org}{\faCloud}}
  My personal containerized platform where I manage all within my \href{https://capass.org}{domain}. Entrypoint and SSL certificates are managed connecting \textit{Cloudflare} to my \textit{Traefik} reverse proxy. Main services are:
  \begin{itemize}
    \item \href{https://git.capass.org}{\textbf{Git server}}: where all my code is stored;
    \item \href{https://cloud.capass.org}{\textbf{Cloud storage}}: dedicated self-hosted cloud storage solution with Nextcloud;
    \item \textbf{Wireguard}: a personal VPN using Wireguard;
    \item (upcoming) portfolio: a repository where all my personal projects are showed in details
  \end{itemize}

  \textbf{Keywords:} Cloud, Docker, Wireguard, VPN, Cloudflare, Traefik

  \subsection{Wellness Nutrition \href{https://github.com/alarmfox/wellness-nutrition}{\faGithub} \href{https://wellnessdemo.capass.org}{\faLaptopCode}}
  A slot management application for a local gym. It allows admins to manage their own users and time slots. Users can book a slot with their personal trainer according to their subscription. The application supports in app and e-mail notifications. The application is self-hosted on a VPS with docker-compose.\\

  \textbf{Keywords:} Cloud, Next.js, Web development, Github Actions, Docker

  \section{Technical skills}
  \subsubsection{Programming}
  C, C++, Go, Rust

  \subsubsection{Scripting}
  Python, Javascript/Typescript, Lua, Bash, SQL

  \subsubsection{Technologies}
  PostgreSQL, MongoDB, Cassandra, Nats, Apache Kafka, ActiveMQ
  
  \subsubsection{Cybersecurity}
  Metasploit, Burpsuite, Qualys, nmap, Nessus, Cisco Umbrella, IPsec, Wireguard

  \subsubsection{Tools}
  Git, Jupyter Notebook, Vivado, Matlab 

  \subsubsection{Markup}
  Latex, HTML, CSS

  \subsubsection{Networking}
  Knowledge of MQTT, Modbus, SNMP, LoRaWAN protocols. Cisco ISE/Packetfence, Cisco IOS/IOX

  \subsubsection{Devops and Cloud}
  Github Actions, Gitlab CI/CD, Docker, Docker Swarm, Kubernetes, Azure, Hetzner

  % \subsubsection{Data analysis}
  % Basic statistical tools: PCA, clustering, linear regression, DoE
\end{document}
